%%%%%%%%%%%%%%%%%%%%%%%%%%%%%%%%%%%%%%%%%%%%%%%%%%%%%%%%%%%%%%%%%%%%%%%%%%%%%%%%
% assignment_template.tex
% A template for assignments
% https://github.com/mhyee/latex-examples/
%%%%%%%%%%%%%%%%%%%%%%%%%%%%%%%%%%%%%%%%%%%%%%%%%%%%%%%%%%%%%%%%%%%%%%%%%%%%%%%%


% LaTeX Preamble
% Load packages and set options as needed
%%%%%%%%%%%%%%%%%%%%%%%%%%%%%%%%%%%%%%%%%%%%%%%%%%%%%%%%%%%%%%%%%%%%%%%%%%%%%%%%

% Set the document class to "article"
% Pass it "letterpaper" option
\documentclass[12pt, letterpaper]{article}

% We don't need the special font encodings, but still
% good practice to include these. See:
%
% http://tex.stackexchange.com/questions/664/why-should-i-use-usepackaget1fontenc
% http://dsanta.users.ch/resources/type1.html
\usepackage[T1]{fontenc}
\usepackage{ae,aecompl}
% http://tex.stackexchange.com/a/44699
% http://tex.stackexchange.com/a/44701
\usepackage[utf8]{inputenc}

% Use Latin Modern, an improved version of the Computer Modern font
\usepackage{lmodern}

% Nicer monospace font, when we use \texttt
\usepackage{inconsolata}

% Code Blocks
\usepackage{listings}

% Set the margins
\newcommand{\margin}{2cm}
\usepackage[top=\margin,right=\margin,left=\margin,bottom=\margin]{geometry}

% Use fancyhdr to define our own headers
\usepackage{fancyhdr}
\setlength{\headheight}{25pt} % Keeps LaTeX happy, takes care of some warnings
\pagestyle{fancy}

% Definitions to fill the header with
% EDIT THESE FIELDS
%%%%%%%%%%%%%%%%%%%%%%%%%%%%%%%%%%%%%%%%%%%%%%%%%%%%%%%%%%%%%%%%%%%%%%%%%%%%%%%%
\newcommand{\course}{CS 486}
\newcommand{\assignment}{Assignment 4}
\newcommand{\nameo}{}
\newcommand{\ido}{}
\newcommand{\namet}{}
\newcommand{\idt}{}
\renewcommand{\date}{\today}
%%%%%%%%%%%%%%%%%%%%%%%%%%%%%%%%%%%%%%%%%%%%%%%%%%%%%%%%%%%%%%%%%%%%%%%%%%%%%%%%

% Now define the header. Make the text bold.
% We'll get something like:
%
% 123456789             LaTeX 101       123456789
% J. Random Student   Assignment N      J. Random Student
% --------------------------------------------------
%
% This layout is pretty simple, and should be enough for an assignment
% If you want more, you can consult the documentation
% http://www.ctan.org/tex-archive/macros/latex/contrib/fancyhdr/fancyhdr.pdf
\lhead{\textbf{\ido\\ \nameo}}
\chead{\textbf{\course\\ \assignment}}
\rhead{\textbf{\idt\\ \namet}}

% Here is an example for customising the numbering
% It changes the first level of numbering to bolded (a), (b), (c), etc
\renewcommand{\theenumi}{\textbf{\alph{enumi}. }}
\renewcommand{\labelenumi}{\theenumi}
% Other options to play with are to change \theenumii, \labelenumii, and enumii for the second level of nesting,
% and so on to \theenumiv, \labelenumiv, and enumiv for the fourth level of nesting.
% The possible formats are \arabic (1, 2...), \alph (a, b...), \Alph (A, B...), \roman (i, ii...), and \Roman (I, II...)

% Begin the actual typesetting, by starting the "document" environment
%%%%%%%%%%%%%%%%%%%%%%%%%%%%%%%%%%%%%%%%%%%%%%%%%%%%%%%%%%%%%%%%%%%%%%%%%%%%%%%%
%----------------------------------------------------------------------------------------
% NAME AND CLASS SECTION
%----------------------------------------------------------------------------------------

\newcommand{\hmwkTitle}{Assignment\ \#4} % Assignment title
\newcommand{\hmwkClass}{CS\ 486 - Fall 2016} % Course/class

\newcommand{\groupMemberOne}{Hanumanth Kumar Jayakumar - 20527136}
\newcommand{\groupMemberTwo}{Changqi Du - 20508921}
\newcommand{\groupMemberThree}{Difei Zhang - 20530592}
\newcommand{\groupMemberFour}{Jichen Zhao - 20453860}

%----------------------------------------------------------------------------------------
% TITLE PAGE
%----------------------------------------------------------------------------------------

\title{
\vspace{2in}
\textmd{\textbf{\hmwkTitle\\ \large\hmwkClass}}\\
\normalsize\vspace{0.1in}\large{\groupMemberOne}\\
\normalsize\vspace{0.1in}\large{\groupMemberTwo}\\
\normalsize\vspace{0.1in}\large{\groupMemberThree}\\
\normalsize\vspace{0.1in}\large{\groupMemberFour}\\
\vspace{3in}
}

\begin{document}

  \maketitle
  \newpage

  \section*{Generated Text Samples \& Evaluation}

  In this document, we will see some samples of text generated after enriching our natural language resources.
  We will also evaluate these texts informally, based on their fluency and interestingness.

  \subsection*{Wehrmacht 18th Army}

  With max graph distance: 1
  \begin{lstlisting}
	Wehrmacht 18th Army was an army. It was led by Georg Von Kuchler. 
	It was a participant of the Siege of Leningrad.
  \end{lstlisting}

  \noindent
  With max graph distance: 2
  \begin{lstlisting}
	Wehrmacht 18th Army was an army. An army is a kind of Unit. 
	It consists of at least two corps. 
	Wehrmacht 18th Army was led by Georg Von Kuchler. 
	Georg Von Kuchler was a general. He served Germany. 
	Wehrmacht 18th Army was a participant of the Siege of Leningrad. 
	The Siege of Leningrad was an engagement. It took place in Leningrad. 
	It was defended by the Allies. It was attacked by the Axis powers. 
	It was won by the Allies. 
	Wehrmacht 16th Army and Wehrmacht 18th Army were participants of it. 
	It ended on 1944-01-27T23:59:59Z. It started on 1941-09-08T00:00:00Z.
  \end{lstlisting}

  \vspace{4mm}

  \noindent
  For this army, we see that with a max graph distance of 1, we have gramatically accurate text.
  It is fluent, albeit with slightly short sentences.
  With a max graph distance of 2, we have more interesting text. It is also gramatically accurate for the most part.
  While it is fluent, it also suffers from the same problem of sentences which are too short.
  In spoken English, we tend to combine short sentences using conjunctions, and their absence can be felt in the above
  two passages of text.


  \subsection*{Siege of Leningrad}

  With max graph distance: 1
  \begin{lstlisting}
	The Siege of Leningrad was an engagement. It took place in Leningrad. 
	It was defended by the Allies. It was attacked by the Axis powers. 
	It was won by the Allies. 
	Wehrmacht 16th Army and Wehrmacht 18th Army were participants of it. 
	It ended on 1944-01-27T23:59:59Z. It started on 1941-09-08T00:00:00Z.
  \end{lstlisting}

  \vspace{4mm}

  \noindent
  For this engagement, we see that we have quite rich information with a max graph distance of 1.
  The text is also gramatically accurate and fluent, albeit with slightly short sentences.
  Since we have some interesting information here, we will not include the sample of graph distance 2.
  The fluency of the text can be improved by combining the start and end dates into one sentence, as well as the
  powers involved.
  The dates used here are too detailed as well, and we should be able to shorten them into dates used in daily life.


  \subsection*{Army}

  With max graph distance: 1
  \begin{lstlisting}
	An army is a kind of Unit. 
	It consists of at least two corps. 
	It was led by only generals.
  \end{lstlisting}

  \vspace{4mm}

  \noindent
  This is an example of sample text generated for a class rather than an individual.
  It is not completely gramatically accurate as evidenced by the last sentence.
  This is one of the problems we faced - how would we remove constraints from the class in its generated text.
  It is interesting in the sense that it conveys some information about an army by virtue of a brief description.


  \subsection*{Battalion}
  
  With max graph distance: 1
  \begin{lstlisting}
	  Battalion isn't a kind of corps, division, squad, 
	  platoon, company, and brigade. 
	  It is a kind of Unit. It was led by exactly one Lieutenant Colonel. 
	  It consists of at least four and at most six companies.
  \end{lstlisting}

  \vspace{4mm}

  \noindent
  This is another example of sample text generated for a class rather than an individual.
  We see a few issues with this text. Firstly, it lists all the classes that it is disjoint with.
  This conveys too much information which is not required. 
  Humans would generally assume that a battalion is not a kind of any of the other units, unless told otherwise.
  This brings up an interesting point in Natural Language Generation.
  We also see the specification that it was led by 'exactly one' lieutenant colonel. 
  This again follows from the above, where we would say 'a' instead of exactly one in spoken or written English.
  The last sentence conveys some interesting information about this text, it gives us a range of companies which make up
  a battalion.
  This could be reworded as 'A battalion consists of between four to six companies'.

  \vspace{4mm}

  \noindent
  From the above sample texts, we can infer that our natural resources enable better text generation for instances,
  as compared to classes.
  In all cases, the generated text is gramatically accurate for the most part.
  One of the shortcomings is that the setences generated are too short.
  Thus, the fluency can be improved by combining short sentences using conjuctions, as we would do in day to day use.

  \vspace{4mm}

  \noindent
  In terms of interestingness, we observe that some samples provide interesting information with a max graph distance of 1,
  while others require a graph distance of 2 to produce interesting information.
  We believe there is a direct correlation between the number of natural language resources we add or improve and the
  interestingness of the text.
  To test this out, we compared two samples of text with a few NL names unlinked, and then linked.
  We observed that the latter produced more interesting text.

  \vspace{4mm}

  \noindent
  Overall, we have some decent natural language resources implemented.
  As always, this can be improved by further enriching our ontology, as well as our Lexicon entries, NL names, and sentence plans.
  In general, Natural Language Generation is hard, but we can solve subsets of problems with current techniques.


\end{document}
