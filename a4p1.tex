%%%%%%%%%%%%%%%%%%%%%%%%%%%%%%%%%%%%%%%%%%%%%%%%%%%%%%%%%%%%%%%%%%%%%%%%%%%%%%%%
% assignment_template.tex
% A template for assignments
% https://github.com/mhyee/latex-examples/
%%%%%%%%%%%%%%%%%%%%%%%%%%%%%%%%%%%%%%%%%%%%%%%%%%%%%%%%%%%%%%%%%%%%%%%%%%%%%%%%


% LaTeX Preamble
% Load packages and set options as needed
%%%%%%%%%%%%%%%%%%%%%%%%%%%%%%%%%%%%%%%%%%%%%%%%%%%%%%%%%%%%%%%%%%%%%%%%%%%%%%%%

% Set the document class to "article"
% Pass it "letterpaper" option
\documentclass[12pt, letterpaper]{article}

% We don't need the special font encodings, but still
% good practice to include these. See:
%
% http://tex.stackexchange.com/questions/664/why-should-i-use-usepackaget1fontenc
% http://dsanta.users.ch/resources/type1.html
\usepackage[T1]{fontenc}
\usepackage{ae,aecompl}
% http://tex.stackexchange.com/a/44699
% http://tex.stackexchange.com/a/44701
\usepackage[utf8]{inputenc}

% Use Latin Modern, an improved version of the Computer Modern font
\usepackage{lmodern}

% Nicer monospace font, when we use \texttt
\usepackage{inconsolata}

% Code Blocks
\usepackage{listings}

% Set the margins
\newcommand{\margin}{2cm}
\usepackage[top=\margin,right=\margin,left=\margin,bottom=\margin]{geometry}

% Use fancyhdr to define our own headers
\usepackage{fancyhdr}
\setlength{\headheight}{25pt} % Keeps LaTeX happy, takes care of some warnings
\pagestyle{fancy}

% Definitions to fill the header with
% EDIT THESE FIELDS
%%%%%%%%%%%%%%%%%%%%%%%%%%%%%%%%%%%%%%%%%%%%%%%%%%%%%%%%%%%%%%%%%%%%%%%%%%%%%%%%
\newcommand{\course}{CS 486}
\newcommand{\assignment}{Assignment 4}
\newcommand{\nameo}{}
\newcommand{\ido}{}
\newcommand{\namet}{}
\newcommand{\idt}{}
\renewcommand{\date}{\today}
%%%%%%%%%%%%%%%%%%%%%%%%%%%%%%%%%%%%%%%%%%%%%%%%%%%%%%%%%%%%%%%%%%%%%%%%%%%%%%%%

% Now define the header. Make the text bold.
% We'll get something like:
%
% 123456789             LaTeX 101       123456789
% J. Random Student   Assignment N      J. Random Student
% --------------------------------------------------
%
% This layout is pretty simple, and should be enough for an assignment
% If you want more, you can consult the documentation
% http://www.ctan.org/tex-archive/macros/latex/contrib/fancyhdr/fancyhdr.pdf
\lhead{\textbf{\ido\\ \nameo}}
\chead{\textbf{\course\\ \assignment}}
\rhead{\textbf{\idt\\ \namet}}

% Here is an example for customising the numbering
% It changes the first level of numbering to bolded (a), (b), (c), etc
\renewcommand{\theenumi}{\textbf{\alph{enumi}. }}
\renewcommand{\labelenumi}{\theenumi}
% Other options to play with are to change \theenumii, \labelenumii, and enumii for the second level of nesting,
% and so on to \theenumiv, \labelenumiv, and enumiv for the fourth level of nesting.
% The possible formats are \arabic (1, 2...), \alph (a, b...), \Alph (A, B...), \roman (i, ii...), and \Roman (I, II...)

% Begin the actual typesetting, by starting the "document" environment
%%%%%%%%%%%%%%%%%%%%%%%%%%%%%%%%%%%%%%%%%%%%%%%%%%%%%%%%%%%%%%%%%%%%%%%%%%%%%%%%
%----------------------------------------------------------------------------------------
% NAME AND CLASS SECTION
%----------------------------------------------------------------------------------------

\newcommand{\hmwkTitle}{Assignment\ \#4} % Assignment title
\newcommand{\hmwkClass}{CS\ 486 - Fall 2016} % Course/class

\newcommand{\groupMemberOne}{Hanumanth Kumar Jayakumar - 20527136}
\newcommand{\groupMemberTwo}{Changqi Du - 20508921}
\newcommand{\groupMemberThree}{Difei Zhang - 20530592}
\newcommand{\groupMemberFour}{Jichen Zhao -20453860}

%----------------------------------------------------------------------------------------
% TITLE PAGE
%----------------------------------------------------------------------------------------

\title{
\vspace{2in}
\textmd{\textbf{\hmwkTitle\\ \large\hmwkClass}}\\
\normalsize\vspace{0.1in}\large{\groupMemberOne}\\
\normalsize\vspace{0.1in}\large{\groupMemberTwo}\\
\normalsize\vspace{0.1in}\large{\groupMemberThree}\\
\normalsize\vspace{0.1in}\large{\groupMemberFour}\\
\vspace{3in}
}

\begin{document}

  \maketitle
  \newpage

  \section*{Experimental Plan}

  	\begin{enumerate}
  		\item
  		We plan to research natural language resource issues. 
      Our goal is to extend our ontology using Lexicon entries, NL Names and Sentence Plans.
  		We would like to explore text generation using NL resources, and the effect of enriching these resources 
      on the generated text.
  		We will be documenting our steps and results throughout this process so that we have sufficient information 
      to make informal evaluations and educated guesses.

  		\item
  		Our primary domain ontology is a knowledge base focused on World War 2.
      It contains information about the different participants and alliances fighting the war.
      We also model different engagements in various locations, and the types of units taking part in these engagements.
      The ontology also contains information about the different types of commanders leading different units, and
      the types of army men taking part in the war.

  		\item
      We do not plan to use any existing ontologies. However we plan to use the sample Harry Potter ontology, and its 
      associated Natural Language resources as a guide while creating our own. 
      This will serve as a benchmark for enriching our own natural language resources. 
      We also plan to use the \textit{'Generating Natural Language Descriptions from OWL Ontologies: the NaturalOWL System'} 
      document to further our understanding of the NaturalOWL System.

  		\item
      We plan to be able to form fluent sentences about different armies, the battles they took part in, and different
      army units.
      If possible, these sentences would be interesting as well, but that also depends on the amount of information
      present in the ontology.
      This will be achieved by creating lexicon entries for the required nouns, verbs and adjectives.
      Following this, we will create NL names for certain lexicon entries, and connect them to classes and individuals.
      Finally, we will create sentence plans, connect them to object properties, and assess the effect of these changes 
      on the generated text.

      The above steps will be carried out for each sentence, or topic that we wish to generate fluent text for.
      During this process, we will constantly compare and iterate on our changes.
      This will enable us to view the results of each change, as well as find the best option out of multiple choices
      of generated text.

  		\item
      We will test the effect of adding more sentence plans and other natural language resources on the generated text.
      Our hypotheses is that an increase in the natural language resources should lead to more fluent and gramatically
      accurate text.    
      These resources should help form proper sentences with pronouns and be somewhat similar to language used in 
      daily life.
  		
  	
  	\end{enumerate}
  
  \pagebreak

  \section*{Description of Parts Implemented}

    We started our experiments with generating text for the Wehermacht 18th Army individual. 
    We found that using a maximum graph distance of 2 generated text with richer content compared to a graph distance of 1.
    One of the problems we faced initially was that inferred object properties did not show up in the generated text. 
    In order to solve this problem, we asserted a couple of previously inferred properties.
    We also had trouble figuring out how to generate text with the correct article before it. 
    We solved this by creating articles in NL names, and connecting these NL names to individuals or classes.
    Also, we figured that it is best practice to have the articles agree with the noun they are referring to in terms of
    plurality, although this is hard to test since in most situations we only have singular nouns.

    \vspace{4mm}

    \noindent
    We followed the plan as described above, beginning with creation of lexicons and NL names.
    After linking each NL name to an individual or class, we tested its effect on text generation.
    This iterative process was useful in debugging, since we forgot to add the connection to a certain class a couple of
    times. (Clicking on the class from the NL names connectiong dialog does not add the connection ! You need to click 
    on the + button !)

    \vspace{4mm}

    \noindent
    Once we added in the required lexicons and NL names for a sentence, we create a sentence plan, and connected the plan
    to an object property. A few times, this gave us some interesting results. 
    For example, if the gender of an inanimate object, was left at the default Masculine/Feminine setting,
    the sentence plan would replace the property owner with \textit{he/she} in the generated text. 
    This resulted in some pretty funny sentences which made no sense at all !

    \vspace{4mm}

    \noindent
    During this process of enriching our natural language resources, we found that we could not use the same name for
    the lexicon entry and the NL name. 
    We also made good use of the 'duplicate' feature, which enabled us to easily replicate similar lexicon entries or
    NL names.
    When playing with the different forms of lexicons such as the singular and plural forms, we found that it is often
    useful to experiment with wildly distinct strings such as "generalZZZ" instead of "generals" to more clearly see that
    the proper form is used.
    Initially, we used strings within Sentence plans for certain verbs which had to be modified from the object property.
    However, we realized that adding those verbs to the lexicon entries, and utilizing the system provided toBEVLE
    would make our NL resources more extensible.
    So, we added in more verbs to our Lexicon, to make it easier to understand and reuse.

    \vspace{4mm}

    \noindent
    During the process of forming accurate sentences for Wehermacht's 18th Army, we found that each successive addition
    to the NL resources, enriched our generated text.
    It added more fluency and grammatical correctness to the text.
    Our ontology was based on World War 2, so we also had to modify the default isASPEN sentence plan, to reflect the
    correct tense in the text generated.

    \vspace{4mm}

    \noindent
    Once we generated text for the Wehermacht 18th Army, we followed the same process as above for the other armies.
    We then moved on to enriching the resources for the different engagement instances since they had more interesting
    information.
    We were able to add a number of sentence plans, lexicons, and NL names in this process.
    We were also able to rearrange our sentence plan from the defualt order, to form more gramatically accurate sentences.
    The \textit{Siege of Leningrad} in particular gave us some really interesting information compared to other instances.

    \pagebreak

    \noindent
    Finally, we added natural language resources for the different types of units.
    This was interesting because we had focused on individuals before, but we were now working with classes.
    Since constraints on these classes were part of the generated text, we tried to add resources which
    would portray them in an accurate manner.
    However, we are still left with some funny sentences, caused by universal and existential constraints.
    Overall, we managed to implement all the parts of the plan as envisioned, and
    learned a lot about language and the difficulties involved in NLG.


\end{document}
