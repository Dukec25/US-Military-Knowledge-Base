%%%%%%%%%%%%%%%%%%%%%%%%%%%%%%%%%%%%%%%%%%%%%%%%%%%%%%%%%%%%%%%%%%%%%%%%%%%%%%%%
% assignment_template.tex
% A template for assignments
% https://github.com/mhyee/latex-examples/
%%%%%%%%%%%%%%%%%%%%%%%%%%%%%%%%%%%%%%%%%%%%%%%%%%%%%%%%%%%%%%%%%%%%%%%%%%%%%%%%


% LaTeX Preamble
% Load packages and set options as needed
%%%%%%%%%%%%%%%%%%%%%%%%%%%%%%%%%%%%%%%%%%%%%%%%%%%%%%%%%%%%%%%%%%%%%%%%%%%%%%%%

% Set the document class to "article"
% Pass it "letterpaper" option
\documentclass[12pt, letterpaper]{article}

% We don't need the special font encodings, but still
% good practice to include these. See:
%
% http://tex.stackexchange.com/questions/664/why-should-i-use-usepackaget1fontenc
% http://dsanta.users.ch/resources/type1.html
\usepackage[T1]{fontenc}
\usepackage{ae,aecompl}
% http://tex.stackexchange.com/a/44699
% http://tex.stackexchange.com/a/44701
\usepackage[utf8]{inputenc}

% Use Latin Modern, an improved version of the Computer Modern font
\usepackage{lmodern}

% Nicer monospace font, when we use \texttt
\usepackage{inconsolata}

% Code Blocks
\usepackage{listings}

% Set the margins
\newcommand{\margin}{2cm}
\usepackage[top=\margin,right=\margin,left=\margin,bottom=\margin]{geometry}

% Use fancyhdr to define our own headers
\usepackage{fancyhdr}
\setlength{\headheight}{25pt} % Keeps LaTeX happy, takes care of some warnings
\pagestyle{fancy}

% Definitions to fill the header with
% EDIT THESE FIELDS
%%%%%%%%%%%%%%%%%%%%%%%%%%%%%%%%%%%%%%%%%%%%%%%%%%%%%%%%%%%%%%%%%%%%%%%%%%%%%%%%
\newcommand{\course}{CS 486}
\newcommand{\assignment}{Assignment 4}
\newcommand{\nameo}{}
\newcommand{\ido}{}
\newcommand{\namet}{}
\newcommand{\idt}{}
\renewcommand{\date}{\today}
%%%%%%%%%%%%%%%%%%%%%%%%%%%%%%%%%%%%%%%%%%%%%%%%%%%%%%%%%%%%%%%%%%%%%%%%%%%%%%%%

% Now define the header. Make the text bold.
% We'll get something like:
%
% 123456789             LaTeX 101       123456789
% J. Random Student   Assignment N      J. Random Student
% --------------------------------------------------
%
% This layout is pretty simple, and should be enough for an assignment
% If you want more, you can consult the documentation
% http://www.ctan.org/tex-archive/macros/latex/contrib/fancyhdr/fancyhdr.pdf
\lhead{\textbf{\ido\\ \nameo}}
\chead{\textbf{\course\\ \assignment}}
\rhead{\textbf{\idt\\ \namet}}

% Here is an example for customising the numbering
% It changes the first level of numbering to bolded (a), (b), (c), etc
\renewcommand{\theenumi}{\textbf{\alph{enumi}. }}
\renewcommand{\labelenumi}{\theenumi}
% Other options to play with are to change \theenumii, \labelenumii, and enumii for the second level of nesting,
% and so on to \theenumiv, \labelenumiv, and enumiv for the fourth level of nesting.
% The possible formats are \arabic (1, 2...), \alph (a, b...), \Alph (A, B...), \roman (i, ii...), and \Roman (I, II...)

% Begin the actual typesetting, by starting the "document" environment
%%%%%%%%%%%%%%%%%%%%%%%%%%%%%%%%%%%%%%%%%%%%%%%%%%%%%%%%%%%%%%%%%%%%%%%%%%%%%%%%
%----------------------------------------------------------------------------------------
% NAME AND CLASS SECTION
%----------------------------------------------------------------------------------------

\newcommand{\hmwkTitle}{Assignment\ \#4} % Assignment title
\newcommand{\hmwkClass}{CS\ 486 - Fall 2016} % Course/class

\newcommand{\groupMemberOne}{Hanumanth Kumar Jayakumar - 20527136}
\newcommand{\groupMemberTwo}{Changqi Du - }
\newcommand{\groupMemberThree}{Difei Zhang - }
\newcommand{\groupMemberFour}{Jason Zhao - }

%----------------------------------------------------------------------------------------
% TITLE PAGE
%----------------------------------------------------------------------------------------

\title{
\vspace{2in}
\textmd{\textbf{\hmwkTitle\\ \large\hmwkClass}}\\
\normalsize\vspace{0.1in}\large{\groupMemberOne}\\
\normalsize\vspace{0.1in}\large{\groupMemberTwo}\\
\normalsize\vspace{0.1in}\large{\groupMemberThree}\\
\normalsize\vspace{0.1in}\large{\groupMemberFour}\\
\vspace{3in}
}

\begin{document}

  \maketitle
  \newpage

  \section*{Experimental Plan}

  	\begin{enumerate}
  		\item
  		We plan to research natural language resource issues. 
      Our goal is to extend our ontology using Leixon entries, NL Names and Sentence Plans.
  		We would like to explore text generation using NL resources, and the effect of of enriching these resources 
      on the generated text.
  		We will be documenting our steps and results throughout this process so that we have sufficient information 
      to make informal evaluations and educated guesses.

  		\item
  		Our primary domain ontology is a knowledge base focused on World War 2.
      It contains information about the different participants and alliances fighting the war.
      We also model different engagements in various locations, and the types of units taking part in these engagements.
      The ontology also contains information about the different types of commanders leading different units, and
      the types of army men taking part in the war.

  		\item
      We do not plan to use any existing ontologies. However we plan to use the sample Natural Language resources as a
      guide while creating our own. This will serve as a benchmark for enriching our resources.

  		\item
      We plan to create the following extensions to the natural language resources.

  		\item
      We will test the effect of adding more sentence plans and other natural language resources on the generated text.
      Our hypotheses is that an increase in the natural language resources should lead to a more fluent and gramatically
      accurate text.    
      These resources should help form proper sentences with pronouns and be similar to language used in daily life.
  		
  	
  	\end{enumerate}
  
  \section*{Description of Parts Implemented}

\end{document}
